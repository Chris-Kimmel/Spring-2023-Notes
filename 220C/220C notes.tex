\documentclass[x11names,reqno,14pt]{extarticle}
\input{preamble}
\usepackage[document]{ragged2e}
\usepackage{epsfig}

\pagestyle{fancy}{
	\fancyhead[L]{Spring 2023}
	\fancyhead[C]{220C - Fields}
	\fancyhead[R]{John White}
  
  \fancyfoot[R]{\footnotesize Page \thepage \ of \pageref{LastPage}}
	\fancyfoot[C]{}
	}
\fancypagestyle{firststyle}{
     \fancyhead[L]{}
     \fancyhead[R]{}
     \fancyhead[C]{}
     \renewcommand{\headrulewidth}{0pt}
	\fancyfoot[R]{\footnotesize Page \thepage \ of \pageref{LastPage}}
}
\newcommand{\pmat}[4]{\begin{pmatrix} #1 & #2 \\ #3 & #4 \end{pmatrix}}
\newcommand{\mk}[1]{\mathfrak{#1}}
\newcommand{\into}{\hookrightarrow}
\newcommand{\onto}{\twoheadrightarrow}
\DeclareMathOperator{\Perm}{Perm}
\DeclareMathOperator{\Gal}{Gal}
\DeclareMathOperator{\Fix}{Fix}
\DeclareMathOperator{\Intermed}{Intermed}
\DeclareMathOperator{\SubGal}{SubGal}
\DeclareMathOperator{\Char}{char}

\title{220C - Fields}
\author{John White}
\date{Spring 2023}


\begin{document}

\section*{Lecture 1, 4/3/23}

\defn

A \underline{field} extension $F \subseteq K$ is a field $F$, which is a subfield of a larger field $K$.

One way to keep track of how these are related is the \underline{degree}, $[K:F]$. This is the dimension of $K$ as a vector space over $F$. 

If this degree is $<\oo$, then we refer to this as a \underline{finite extension} (we of course do not mean that they are finite as sets)

If $S \subseteq K$, then $F(S)$ is the subfield of $K$ given by $F \cup S$.

$F[S]$ is the sub-\textit{ring} of $K$ generated by $F \cup S$. These are different in general!

If $S = \{a_1,\dots, a_n\}$, we use $F(a_1, \dots, a_n)$ and $F[a_1,\dots, a_n]$ to denote $F(S)/F[S]$. 

If the extension has the form $F[a]$ for some element $a$, then this is called a \underline{simple extension}. Here, $a$ is called a \underline{primitive element}.

An extension $F \subseteq K$ is called \underline{algebraic} if every $k \in K$ is algebraic over $F$, meaning is the root of some polynomial in $F[x]$

\exm\,

\begin{itemize}

\item $Q \subseteq \R$. This is an infinite extension. Further, it is not an algebraic extension. The hard way to show this is to demonstrate that some element of $\R$ is not algebraic. For example, $\pi,e$ are real, but transcendental over the rationals. 

The easy way is by a simple cardinality argument: Because $\Q$ is countable, $\bar{\Q}$ is, but $\R$ is not

\item $\R\subseteq\C$. This is a finite extension. In fact, it is a simple extension, with primitive $i$.

\item $\Q\subseteq\Q(\sqrt{5})$. This is algebraic. Of course, $\sqrt{5}$ is a root of $x^5 - 1$, but what about the other elements of $\Q(\sqrt{5})$? 

Consider $\{a + b\sqrt{5} \mid a, b \in \Q\}$. This is a subset of $\Q(\sqrt{5})$, a subring, and a subfield: indeed, consider $\frac{1}{a + b\sqrt{5}}$. The ``typical high school trick" is to multiply by the conjugate:
\[
\frac{1}{a + b\sqrt{5}}\frac{a - b\sqrt{5}}{a - b\sqrt{5}} = \frac{a - b\sqrt{5}}{a^2 - 5b^2}
\]
So $\Q(\sqrt{5}) = \{a + b\sqrt{5} \mid a, b \in \Q\}$, as this is a subfield of $Q(\sqrt{5})$ which contains $\sqrt{5}$, so must contain $\Q(\sqrt{5})$. That is, 
\[
\Q(\sqrt{5}) = \{a + b\sqrt{5}\mid a, b \in \Q\} = \Q[\sqrt{5}]
\]
It is easy to see that $[\Q(\sqrt{5}):\Q] = 2$

\end{itemize}

Let $F \subseteq K$ be a field extension, and consider $F[a_1,\dots, a_n]$. 

There exists an evaluation map $\varepsilon:F[X_1,\dots, X_n] \to K$, given by $\varepsilon(f) = f(a_1, \dots, a_n)$. $\varepsilon$ is a ring homomorphism, so $\Im(\varepsilon)$ is a subring of $K$. We have $F[a_1,\dots, a_n] = \Im\varepsilon$

$F(a_1,\dots, a_n)$ is a quotient field for the ring $F[a_1,\dots, a_n]$

Let $F$ be a field, $x, y$ be indeterminants which are independent over $F$. Let $L = F(y)[x]/\langle x^2 - y \rangle$. 

We can check that $x^2 - y$ is irreducible in $F(y)[x]$ because it is quadratic, and $y$ has no square roots. 

So because this is irreducible, $L$ is a field.

In particular, $F(y)$ embeds in $L$ via the natural map $F(y) \into F(y)[x] \onto L$. So $F(y) \subseteq L$. This is a degree two extension of $F(y)$.

\prop

If $[K:F]<\oo$, then $F \subseteq K$ is an algebraic extension.

\proof

Let $n = [K:F]$, and let $a \in K$. Look at $1, a, a^2, \dots, a^n$. This is $n + 1$ elements in $K$, so they must be linearly independent over $F$. So there exists $c_0,c_1,\dots,c_n$, not all zero, such that $\sum_{i=0}^nc_ia^i = 0$. Then $f = \sum_{i=0}^nc_ix^i\in F[x]$ is a polynomial to which $a$ is a solution, so $a$ is algebraic.

\qed

\thm (I)

Let $F \subseteq K$ be a field extension, $a \in K$. Then The Following Are Equivalent (TFAE):

\begin{enumerate}

\item $a$ is algebraic over $F$

\item $\dim_FF[a]<\oo$

\item $[F(a):F] < \oo$

\item $F(a) = F[a]$

\end{enumerate}

\proof

Notice that $3\Rightarrow 2$ are really saying the same thing. Further, $2 + 4 \Rightarrow 3$. So if we can connect 1, 2, 4, then 3 will come along for the ride. Therefore, it is enough to show that $1, 2, 4$ are equivalent.

\subsection*{$1\Rightarrow2$}

There exists a nonzero $f \in F[x]$ such that $f(a) = 0$. $f = \sum_{i=0}^nc_ix^i$, where $c_n\neq0$. So $\sum_{i=0}^nc_ia^i = 0$, and so $a^n = \sum_{i=0}^{n - 1}d_ia^i$, with $d_i \in F$ new coefficients. 

Set $V = \sum_{i=0}^{n - 1}Fa^i$. Then $a^n \in V$. So
\begin{align*}
a^{n + 1} & = \sum_{i=0}^{n - 1}d_ia^{i + 1} \\
			 & = \sum_{j=1}^{n - 1}d_{j - 1}a^j + d_{n - 1}a^n
\end{align*}
But $d_{n - 1}a^n = \sum_{i=0}^{n - 1}d_{n - 1}d_ia^i$. 

Induction gets us that $a^j \in V$ for all $j \geq 0$.

So $V$ is closed under multiplication, hence a subring of $K$.

So $V = F[a]$. Note $\dim_FF[a] = \dim_FV \leq n$, because we used $n$ elements to span in the first place. 

\subsection*{$2\Rightarrow4$}

It will be enough to show $F[a]$ is a field. 

Let $x \in F[a]$, $x\neq0$. Define a map $\mu_x:F[a]\to F[a]$, given by $\mu_x(y) = xy$. This is $F$-linear, and $\ker\mu_x = 0$. We have an injective linear transformation from a finite dimensional vector space to itself, so it has to be an isomorphism onto its image. So there exists $x' \in F[a]$ so that $\mu_x(x') = 1$, so $x$ is invertible.

\section*{Lecture 2, 4/5/23}

We continue the proof.

\subsection*{$4\Rightarrow1$}

If $A = 0$, we are done. If $A \neq 0$, then $\frac{1}{a} \in F(a) = F[a]$. 

So $\frac{1}{a} = \sum_{i=1}^mc_ia^i$ where each $c_i \in F$. Note $1 = \sum_{i=0}^mc_ia^{i + 1}$, so $a$ is a root of $-1 + \sum_{i=0}^mc_ix^{i + 1} = 0$

Thus $a$ is algebraic over $F$. 

\qed

\thm 

Assume $a$ is algebraic over $K$.
\begin{enumerate}[label=(\roman*)]

\item There exists a unique monic polynomial $p \in F[x]$ such that $p(a) = 0$ with minimal degree. We call this the \underline{minimal polynomial} for $a$ over $F$, and write $p_{a, F}$. 

\item $p$ is irreducible.

\item If $g \in F[x]$, $g(a) = 0$, then $p \mid g$ in $F[x]$.

\item $[F(a):F] = \deg p$

\item If $n = \deg p$, then $(1, a, a^2, \dots, a^{n + 1})$ is a basis for $F(a)$ over $F$. 

\item Let $\varepsilon:F[x]\to K, \varepsilon(f) = f(a)$. This induces an isomorphism of rings $\bar{\varepsilon}:\frac{F[x]}{\langle p \rangle} \to F(a), \bar{\varepsilon}(f + \langle p \rangle) = f(a)$

\end{enumerate}

\proof\,

\begin{enumerate}[label=(\roman*)]

\item Since $a$ is algebraic over $F$, there exists $f \in F[x]$ such that $f(a) = 0$. Note that we can divide by the leading coefficient to make $f$ monic with $a$ as a root. Find minimal polynomial of this form, and call it $p$. 

Uniqueness: Suppose $p' \in F[x]$ is monic, $p'(a) = 0$ minimal. Then $(p - p')(a) = 0$. Since $\deg(p - p') < \deg p$, if $p - p' \neq 0$, we have found a monic polynomial with smaller degree than $p$ with $a$ as a root. Contradiction

\item Let $\varepsilon:F[x]\to F(a) = F[a]$ be the evaluation map. $\varepsilon$ induces $\bar{\varepsilon}:\frac{F[x]}{\ker\varepsilon} \to F(a)$. Note $\ker\varepsilon=0$. Since $F[x]$ is a PID, $\ker\varepsilon = \langle q \rangle$ where $0 \neq q \in F[x]$. Without loss of generality, assume $q$ is monic. We know
\begin{itemize}

\item $q$ is irreducible
\item $q(a) = 0$
\item When $g \in F[x], g(a) = 0$, then $q \mid g$

\end{itemize}

Thus, if $g \neq 0$, $\deg (q) \leq \deg (g)$. This implies that $q = p$.

\item See above 

\item $\bar{\varepsilon}$ is also an isomorphism of vectors over $F$. Exercise: If $x \in X + \langle p \rangle$, then $(1, x, x^2, \dots, x^{n + 1})$ is a basis for $\frac{F[x]}{\langle p \rangle}$. Thus $(1, a, a^2, \dots, a^{n - 1})$ is a basis for $F(a)$

Furthermore, $[F(a):F] = n = \deg p$

\qed

Let $F \leq K$ be a field extension, $a \in K$ algebraic over $F$. If $p \in F[x]$ is monic and irreducible with $p(a) = 0$, then $p = p_{a, F}$

\item See iv

\item See ii


\end{enumerate}


\exm

Let $a = \sqrt[4]{5} \in \R_{>0}, p = X^4 - 5 \in \Q[x]$. Since $p$ is irreducible over $\Q[x]$, $p = p_{a, F}$.

Note that $p$ is reducible over $\Q(\sqrt{5})[x]$. In fact, $p_{a, \Q[\sqrt{5}]} = x^2 - \sqrt{5}$. We have the tower of fields:
\[
\begin{tikzcd}
\Q(a) \ar[d, dash] \ar[dd, bend right = 60, "4"'] \\
\Q(\sqrt{5}) \ar[d, dash, "2"] \\
\Q \\
\end{tikzcd}
\]

Let $F \subseteq K \subseteq L$ be a tower of fields. If $a \in L$ is algebraic in $F$, then $a$ is also algebraic over $K$. Furthermore, $p_{a, K} \mid p_{a, F}$ in $K[x]$. 

\prop

If $f \in F[x]$ is a nonzero polynomial of degree $n$, then $f$ has at most $n$ roots in $n$. 

\proof

By induction.

$n = 0:$ trivial. 

$n > 0:$ if there are no roots, we're okay. 

Otherwise, there exists $a \in F$ such that $f(a) = 0$. So $f = (x-alg$, for some $g \in F(x)$. 

$g \neq0, \deg g = n - 1$. Thus $g$ has $\leq n -1 $ roots in $F$. 

Since $\{$roots of $f\} = \{a\} \cup \{$roots of $g\}$, there are $\leq n$ roots of $f$. 

Let $F \subseteq K$ be a field extension. Let $\mc{A} = \{a\in K, a$ algebraic over $F\}$. 

If $F$ is infinite, then $|\mc{A}| = |F|$. If $F$ is finite, $|\mc{A}|$ is countable. 

Let $\mbb{A}$ denote the complex numbers which are algebraic over $\Q$. Note $|\mbb{A}| = |\Q| = \aleph_0$

\section*{Lecture 3, 4/7/23}

\thm (Tower rule)

Let $F \subseteq K \subseteq L$ be a tower of fields. Then $[K:F][L:K] = [L:F]$. 

\proof

If $[K:F]=\oo$ or $[L:K]=\oo$, we are done. 

So assume $[K:F] = m, [L:K] = n, m, n < \oo$. 

Let $\{b_1, \dots, b_n\}$ be a basis for $L$ over $K$, and let $\{a_1, \dots, a_m\}$ be a basis for $K$ over $F$. 

We claim $\{a_ib_j \mid 1 \leq i \leq m, 1 \leq j \leq n\}$ is a basis for $L$ over $F$. 

We check it spans: choose $x\in L$. Note $x = \sum_{j=1}^nu_jb_j,$ where each $u_j \in K$. Each $u_j = \sum_{i=1}^mv_{ij}a_i$, where each $v_{ij} \in F$. Thus $x = \sum_{j=1}^n\sum_{i=1}^mv_{ij}a_ib_j$

Linear independence: suppose $\sum_{i=1}^m \underbrace{\sum_{j=1}^nv_{ij}a_i}_{\in K}b_j = 0$. 

Thus $\sum_{j=1}^nv_{ij}a_i = 0$. Thus all $v_{ij} = 0$.

\qed

\cor

Let $F \subseteq K$ be a field extension, $a_1, \dots, a_n \in K$ all algebraic over $F$. Then $F[a_1,\dots, a_n] = F(a_1,\dots, a_n)$ and $[F(a_1, \dots, a_n):F] < \oo$.

\cor Let $F \subseteq K$ be a field extension.

\begin{enumerate}[label=(\alph*)]

\item If $a, b \in K$ are algebraic over $F$, then $[F(a):F] \mid [F(a, b): F], [F(b):F] \mid [F(a, b): F], pF(a, b): F] \leq [F(a): F],[F(b):F]$

\item $\{a\in K \mid a$ algebraic over $F\}$ is a subfield of $K$.

\item If $S \subseteq K$ is a set of elements algebraic over $F$, then $F(S)$ is algebraic over $F$. 

\item Say $K \leq L$ is a field extension. Then $L$ is algebraic over $F$ if and only if $L$ is algebraic over $K$ and $K$ is algebraic over $F$.

\end{enumerate}

\proof

\qed

\defn 

Let $F \subseteq K$ be a field extension. An \underline{$F$-automorphism of $K$} is any isomorphism $\phi:K\to K$ such that $\phi|_F = \Id_F$. 

The \underline{Galois group} of $K$ over $F$ is $G(K:F) = \{F-$automorphisms of $K\}$

\prop

Let $F \subseteq K$ be a field extension, $\phi\in G(K:F)$, $f$ a polynomial in $F(x)$. Then $\phi$ permutes $\underbrace{\{a\in K \mid f(a) = 0\}}_{R_f}$. 

\proof:

Let $f = \sum_{i=1}^nc_ix^i$, wheree $c_i \in F$. 

Choose $a \in K$. $\phi(f(a)) = \phi(\sum_{i=1}^nc_ia^i) = \sum_{i=1}^nc_i\phi(a_i) = f(\phi(a))$. 

Thus $a \in R_f \iff \phi(a) \in R_f$. 

So $\phi$ restricts mto an injective map $R_f \to R_f$. Thus there is a bijection.

\qed

\exm

$\R\subseteq\C$. $\phi\in G(\C:\R)$ must permute roots of $x^2 + 1$, so $\phi(i) = \pm i$. $\phi$ is $\R$-linear and $(1, i)$ is a basis for $\C$ over $\R$. 

Thus $G(\C:\R) = \{\Id, \bar{-}\}$

\exm

Let $F \subseteq K$ be a field extension of degree 2, $\operatorname{char} F \neq 2$. 

Chose $a \in K \setminus F$. 
Note $F[a] = F(a) = K$. So $a$ is algebraic over $F$, $\deg p_{a, F} = 2$. We know $p_{\alpha, F} = X^2 + bX + c$. 

By quadratic formula, 
\[
a = \frac{-b\pm\sqrt{b^2-4c}}{2}
\]
i.e. there is some $r\in K$ such that $r^2 = b^2 - 4ac$ and $a = \frac{-b + r}{2}$.

$\frac{-b-r}{2}$ is another root of $p_{\alpha,F}$. Since $a\not\in F, r\not\in F$. So $r\neq 0$, i.e. $r\neq -r$. 

Thurs $F[r] = F(r) = K$. So $p_{\alpha,F} = X^2 - (b^2 - 4c)$. 

If $\phi \in \Gal(K:F)$, $\phi(r) = \pm r$. 

Thus $|\Gal(K:F)| \leq 2$. 

Exercise: $|\Gal(K:F)| = 2$. 

\exm See chapter 2: there exists $F \subseteq K, [K:F] = 2, \Char F = 2, \Gal(K:F) = \{e\}$. 

\exm

$F = \Q(\zeta) \subseteq K = \Q(\zeta, a), \zeta = e^{\frac{2\pi i}{3}}, a = \sqrt[4]{5}\in \R$. 

$\zeta$ is a root of $X^3 -1 = (X - 1)(X^2 + X + 1)$. 

Thus $p_{\zeta,\Q}\mid X^2 + X +1$. Since $\zeta\not\in\Q,\deg p_{\alpha, \Q}\mid X^2 + X + 1$. 

Since $\zeta\not\in\Q,\deg p_{\zeta,\Q}>1$. Thus $p_{\zeta,\Q} = X^2 + X + 1, [F:\Q] = 2$

Now note $a$ is a root of $X^3 - 5$, which is irreducible. 

Thus $p_{a,\Q} = X^3 - 5, [\Q(a):\Q] = 3$. 

\[
\begin{tikzcd}
& \ar[dl, "3"', dash]\ar[dr, "2", dash] \ar[dd, dash, "6"]K = Q(\zeta, a)\not\subseteq\R & \\
F = \Q(\zeta)\ar[dr, dash, "2"'] & & \Q(a)\subseteq \R\ar[dl, "3", dash] \\
& \Q & 
\end{tikzcd}
\]

\section*{Lecture 4, 4/10/23}

\defn

Let $F \subseteq K$ be a field extension, $F[X]$ a polynomial ring. $f \in F[X]$ \underline{splits over $K$} if $f = a_0(X - a_1)(X - a_2)\cdots(X - a_n)$ for $a_i \in K$. 
4
\defn A \underline{splitting field} for $S \subseteq F[X]$ over $F$ is a field $K$ containing $F$ such that all $f \in S$ split over $K$, and $K$ is minimal. In other words, if $F \subseteq E \subseteq K$, and for all $f \in S, f$ splits over $E$, then $E = K$.

\defn

$F$ is \underline{algebraically closed} if every $f \in F[X]$ splits over $F$.

\defn

An \underline{algebraic closure of $F$} is a field extension $K$ containing $F$ such that $F \subseteq K$ is algebraic and $K$ is algebraically closed. 

\thm (Fundamental theorem of algebra)

$\C$ is algebraically closed. 

$\C$ is an algebraic closure of $\R$, and is thus a splitting field for $X^2 + 1$ over $\R$. 

Another splitting field for $X^2 + 1$ over $\Q$ is $\Q(i)$. 

A splittinng field for $X^3 - 2$ over $\Q$ is $\Q(a,\zeta)$, $a = \sqrt[3]{2}, \zeta$ a root of $X^2 + X + 1$. 

$X^3 - 2 = (X - a)(X - a\zeta)(X - a\zeta^2)$

\[
\begin{tikzcd}
\Q(a, \zeta) \ar[d, dash, "6"] & \Q(\zeta)(a) \ar[d, dash, "3"'] \\
\Q & \Q(\zeta) \ar[d, dash, "2"'] \\
& \Q \\
\end{tikzcd}
\]

Let $F \subseteq K$ be3 a field extension, $F[X]$ a polynomial ring. Say $K$ is a splitting field for $S\subseteq F[X]$ over $F$. Define
\[
R \eqdef \{a\in K \mid f(a) = 0\text{ for some }f\in S\}
\]
Then $K = F(R)$. So $K$ is algebraic over $F$. 

\claim Suppose $K$ is a splitting field over $F$ for some non-constant $f \in F[X]$. Let $a_1, \dots, a_r$ be the roots of $f \in K$. Then we know $K = F(a_1,\dots, a_r)$. Wee claim that $[K:F] \leq (\deg f)!$. 

\proof 

Let $n = \deg f$. We know $p_{a, F}\mid f$, so $[F(a_1) \mid F] = \deg p_{a_1, F} \leq \deg f = n$. 

So $f = (X - a_1)g$, where $\deg g = n - 1$. 

The roots of $g$ are $\subseteq\{a_1,\dots, a_r\}$. Thus $K = F(a_1)(a_1,\dots,a_r)$. 

By induction, $[K:F(a_1)]\leq (n - 1)!$. 

By the tower rule, $[K:F]\leq n!$. 

\qed

\claim Suppose $K$ is algebraically closed. Take $L$ to be the algebraic closure of $F$ in $K$. Then $L$ is algebraically closed. 

\proof

If $f \in L[X]$ is not constant, it has a root $a \in K$. 

$L \subseteq L(a)$ is algebraic, $F \subseteq L$ is algebraic, so $F \subseteq L$ is algebraic, thus $L(a) \subseteq L$. 

This implies that $f$ has a root in $L$. 

\qed

Say $R,S$ are rings, $R[X],S[X]$ polynomial rings, $\phi:R\to S$ a ring homomorphism. Then there exists a unique ring homomorphism $\tilde{\phi}:R[X]\to S[X]$ such that $\tilde{\phi}|_R = \phi$ and $\tilde{\phi}(X) = X$. 

Formula:
\[
\tilde{\phi}\left(\sum_{i=0}^d r_i X^i\right) = \sum_{i=0}^d\phi(r_i) X^i
\]

\thm[Kronecher's Theorem]

Let $F$ be a field, $f \in F[X]$ a non-constant polynomial. Then there exists a field extension $F \subseteq K$ such that $f$ has a root in $K$, and $[K:F] \leq \deg f$. 

\proof

Let $p$ be some irreducible factor of $f$. 

Define $L = F[X]/\langle p \rangle$, which is a field. 

Let $\bar{h} = h + \langle p \rangle$ for $h \in F[X]$. 

Define $\phi:F\to L$ by $\phi(c) = \bar{c}$. We have $\tilde{\phi}:F[X]\to L[X]$. 

We claim $\bar{X}$ is a root of $\tilde{\phi}(f)$.

$p = \sum_{i=0}^n c_i X^i$, $c_i \in F$. 

Then $\tilde{\phi}(p) = \sum_{i=0}^m\bar{c_i}X^i$

\begin{align*}
\tilde{\phi}(p)(\bar{X}) & = \sum_{i=0}^m\bar{c_i}\bar{X^i} \\
								& = \overline{\sum_{i=0}^m c_i X^i}\\
								& = \bar{p} \\ & = \bar{0} \\
\end{align*}
Thus $\bar{X} \in L$ is a root of $\tilde{\phi}(f)$.

\section*{Lecture 5, 4/12/23}

We continue the proof. 

Choose a set $U$ disjoint from $F$ with $|U| = |L\setminus\phi(F)|$. Say $\beta:U\to L\setminus\phi(F)$ is a bijection. 

Extend to a bijection $\beta:F\coprod U \to L$ such that $\beta(a) = \phi(a)$ for all $a \in F$. 

Define $+,\cdot$ on $F \coprod U$ by 
\begin{align*}
- a + b & = \beta^{-1}(\beta(a) + \beta(b)) \\
-a\cdot b & = \beta^{-1}(\beta(a)\beta(b)) \\
\end{align*}
we need to check new $+,\cdot$ agree with $OG$ on $F$. We also need to check that $F \coprod U$ is a field. We will do this later. 

Define $K = F \coprod U$. Note $F$ is a subfield of $K$. $\beta:K\to L$ is an isomorphism,$\beta|_F = \phi$. 

Define $p = \sum_{i=0}^mc_iX^i, c_i \in F$. 
\begin{align*}
0 & = \tilde{\phi}(p)(\bar{X}) \\
& = \sum_{i=0}^m\phi(c_i)\bar{X^i} \\
& = \sum_{i=0}^m\beta(c_i)\beta(\beta^{-1}(\bar{X}))^i \\
& = \beta\left(\sum_{i=0}^mc_i\beta^{-1}(\bar{X})^i\right)\\
& = \beta(p(\beta^{-1}(\bar{X})) \\
\end{align*}
thus $p(\beta^{-1}(\bar{X})) = 0$. 

$[K:F] = \dim_FK = \dim_FL = \deg p \leq \deg f$. 

\qed

\subsection*{\underline{Ordinals}}

I'm not typing all this up i don't get it











\section*{Lecture 6, 4/14/23}

\lem[Extension Lemma]

Let $F_1\subseteq K_1, F_2\subseteq K_2$ be field extensions, $\phi:F_1\to F_2$ an isomorphism. Let $F_2[X]$ be a polynomial ring, $f_1 \in F_1[X]$ irreducible, $f_2 = \tilde{\phi}(f_1)$, $a_i \in K_i$ a root of $f_i$. Then $\phi$ extends to an isomorphism $\theta:F_1(a_1) \to F_2(a_2)$ such that $\theta(a_1) = a_2$. 
\[
\begin{tikzcd}
K_1 \ar[d, dash] & K_2 \ar[d, dash] \\
F(a_1) \ar[d,dash]\ar[r, dotted, "\exists\theta"] & \ar[d,dash]F(a_2) \\
F_1 \ar[r, "\theta"] & F_2 \\
\end{tikzcd}
\]
The proof is in the form of this diagram, I guess:
\proof
\[
\begin{tikzcd}
F_1[X]\ar[dr, "q_1"] \ar[rrr, "\tilde{\phi}", "\cong"'] &  &  & F_2[X] \ar[dl, "q_2"']\\
 & F_1[X]/\langle f_1 \rangle \ar[d, "p_1"', "\cong"] \ar[r, "\psi", "\cong"'] & F_2[X]/\langle f_2 \rangle \ar[d, "p_2"', "\cong"] &  \\
 & F_1(a_1) \ar[r, "p_2 \circ \psi \circ p_1^{-1}", dotted] & F_2(a_2) &  \\
F_1 \ar[uuu, "\subseteq"] \ar[rrr, "\theta", "\cong"'] &  & & F_2[X] \ar[uuu, "\subseteq"'] \\
\end{tikzcd}
\]
\qed

\thm Let $F_1, F_2$ be fields, $F_2[X]$ a polynomial ring, $S_i \subseteq F_i[X]$, and let $\phi:F_1\to F_2$ be an isomorphism, where $\phi_1(S_1) = S_2$. Let $K_i$ be a splitting field of $S_i$ over $F_i$. Then $\phi$ extends to an isomorphism $K_1 \to K_2$. 
\[
\begin{tikzcd}
K_1 \ar[r, "\cong"] & K_2 \\
F_1 \ar[u, "\subseteq"] \ar[r, "\phi"] & F_2 \ar[u, "\subseteq"']
\end{tikzcd}
\]

\proof

Set $\ms{M}= \{(L,\theta) \mid F_1 \subseteq L \subseteq K_1$ a tower, $\theta:L\to K_2$ a homomorphism extending $\phi\}$. 

Define $(L_1,\theta_1) \leq (L_2, \theta_2)$ iff $L_1 \subseteq L_2$, $\theta_1 \subseteq \theta_2$

\begin{itemize}

\item $\leq$ is a partial ordering of $\ms{M}$. 

\item $(F_1, \theta) \in \ms{M}$. 

\item If $\{(L_i,\theta_i)\}_{i\in I}$ is a nonempty chain on $\ms{M}$, then 
\[
\left(\bigcup_{i\in I} L_i, \bigcup_{i\in I}\theta_i\right) \in \ms{M}
\]
is an upper bound for the chain. 

\end{itemize}

By Zorn's lemma, there exists a maximal $(M, \psi) \in \ms{M}$
\[
\begin{tikzcd}
K_1 \ar[r, "\cong"] \ar[d, dash] & K_2 \ar[d, dash] \\
M \ar[r, "\psi","\cong"'] & M' = \psi(M) \\
F_1 \ar[u, "\subseteq"] \ar[r, "\phi"] & F_2 \ar[u, "\subseteq"']
\end{tikzcd}
\]

Towards contradiction, suppose $M \subsetneq K_1$

Then there exists some $g \in S_1$ which does not split over $M$. In $M[X]$, there exists an irreducible factor $f\mid g$ such that $\deg f \geq 2$

$g$ splits over $K$, so $f$ splits over $K$. 

Thus $f$ has a root $a_1 \in K_1$. 

$\tilde{\psi}(f)\in M'$ is irreducible, has a root $a_2 \in K_2$, because $\tilde{\psi}(f) \mid \tilde{\psi}(g) = \tilde{\phi}(g)$ in $M'[X]$. 

By Extension Lemma, $\psi$ extends to an isomorphism $\theta:M(a_1) \to M'(a_2)$.

Now $(M(a_2),\theta) \in \ms{M}$, contradicting that $(M,\psi)$ is the maximal element. 

\qed

\cor

Let $F_1, F_2$ be fields, $\phi:F_1\to F_2$ an isomorphism, $K_i$ an algebraic closure of $F_i$. Then $\phi$ extends to an isomorphism $K_1 \to K_2$. 

\proof

Observe that $K_i$ is a splitting field for $S_i = F_i[X]$ over $F_i$. $\tilde{\phi}(S_1) = S_2$. Apply the previous theorem.

\qed

\exm $K = \Q(a, \zeta) \subset \C$, $a = \sqrt[3]{5}$, $\zeta = $ a root of $X^2 + X + 1$. 

Roots of $X^3 - 5: a, a\zeta, a\zeta^2$

Roots of $X^2 + X + 1: \zeta, \zeta^2$

\[
K = \Q(a, \zeta) = \Q(a\zeta,\zeta) = \Q(a\zeta^2,\zeta)
\]

\[
\begin{tikzcd}
K = \Q(a, \zeta) \ar[d, dash, "2"]\ar[r, equals] & \ar[d, dash, "2"]\Q(a\zeta,\zeta) \ar[r, equals] &\ar[d, dash, "2"] \Q(a\zeta^2, \zeta) \\
\Q(a\zeta)\ar[dr, "3", dash] & \Q(a\zeta) \ar[d, "3", dash] &\Q(a\zeta^2)\ar[dl, "3"', dash] \\
& \Q &  \\
\end{tikzcd}
\]

$i = 0, 1, 2$, there exists $\Q$-automorphism $\phi_i:\Q(a)\to\Q(a, \zeta^i)$ such that $\phi_i(a) = a\zeta^i$. 

$\phi_i$ extends to $\phi_{ij}:K \to K$ such that $\phi_{ij}(\zeta) = \zeta^j$






\section*{Lecture 7, 4/17/23}

Exercise: $\Gal(\Q(\sqrt{3},\sqrt{5}):\Q) \cong \Z_2\oplus\Z_2$. 

Let $a = e^{\frac25 \pi i}$. $\Gal(\Q(a):\Q) \cong \Z_4$
\,

\defn\,
\begin{itemize}

\item Let $H \leq \Gal(K:F)$. Then define
\[
\Fix_K(H) \eqdef \{k \in K \mid h(k) = k\text{ for all }h \in H\}
\]
This will be an intermediate field.

\item Let $F \subseteq K$ be a field extension. Then we define
\[
\Intermed(F\subseteq K) \eqdef \{\text{fields }L \mid F \subseteq L \subseteq K\}
\]
\[
\SubGal(K\supseteq F) \eqdef \{\text{subgroups of }\Gal(K:F)\}
\]
We have $\ms{F}:\SubGal\to\Intermed, H\mapsto \Fix_K(H)$, $\ms{G}: \Intermed\to\SubGal, L \mapsto \Gal(K:L)$. 

\item $L \in \Intermed$ is \underline{closed} if $\ms{F}(\ms{G}(L)) = L$. It is \underline{stable} if $\phi(L) \subseteq L$ for all $\phi \in \Gal(K:F)$. $H \in \SubGal$ is \underline{closed} if $\ms{G}(\ms{F}(H)) = H$

\end{itemize}

If $L$ is stable, $\phi(L)\subseteq L, \phi^{-1}(L)\subset L$ for all $\phi\in\Gal(K:F)$, so $\phi(L) = L$.

We have restriction map $\rho:\Gal(K:F)\to\Gal(L:F), \rho(\phi) = \phi|_L$. We have $\ker(\rho) = \Gal(K:L)$

\prop\,

\begin{enumerate}

\item $\ms{F}$ and $\ms{G}$ preserve inclusions.

\item 

\begin{enumerate}[label=(\alph*)] 

\item $L \subseteq \ms{F}\ms{G}(L)$ and $\ms{G}(L) = \ms{F}\ms{F}\ms{G}(L)$ for all $L \in \Intermed$

\item $H \subseteq \ms{G}\ms{F}(H)$ and $\ms{F}(H) = \ms{F}\ms{G}\ms{F}(H)$ for all $H \in \SubGal$

\end{enumerate}

\item

\begin{enumerate}[label=(\alph*)]

\item $L \in \Intermed$ is closed iff $L \in \Im(\ms{F})$

\item $H \in \SubGal$ is closed iff $H \in \Im(\ms{G})$

\end{enumerate}

\item $\ms{G}$ and $\ms{F}$ restrict to inverse bijections.

\end{enumerate}

So there is a bijection between $\{\text{closed }L \in \Intermed\}$ and $\{$closed subgroups$\}$

\prop\,

\begin{enumerate}

\item Let $E \subseteq L$ in $\Intermed$ such that $[L:E] < \oo$. Then
\begin{enumerate}[label=(\alph*)]

\item $[\ms{G}(E): \ms{G}(L)] \subseteq [L:E]$ 

\item If $E$ is closed, then so is $L$, and $a$ becomes an equality. 

\end{enumerate}

\item Let $H \subseteq J$ in $\SubGal$ such that $[J:H]<\oo$. 

\begin{enumerate}[label=(\alph*)]

\item $[\ms{F}(H):\ms{F}(G)] \leq [J:H]$

\item If $H$ is closed, so is $J$, and $a$ is an equality

\end{enumerate}

\end{enumerate}

\proof\,

\begin{enumerate}

\item 

\begin{enumerate}[label=(\alph*)]

\item $L = E(a_1, \dots, a_m)$. Induct on $m$. 

For $m = 1$, $L = E(a)$. Since $[L:E] < \oo$, $a$ is algebraic over $E$. So it has minimal polynomial $p = p_{a,E}$. 

The number of roots of $p$ in $K$ is less than or equal to $\deg p = [L:E]$. 

We need to show $[\ms{G}(E), \ms{G}(L)] \leq$ number of roots of $p$ in $K$. 

$H = \ms{G}(L), J = \ms{G}(E)$. 

$\phi_1(H), \cdots, \phi_n(H)$ are distinct left cosets of $H$ in $J$. 

for $i \neq j$, $\phi_i(H) \neq \phi_j(H)$, so $\phi_j^{-1}\phi_i(a) \neq a$, so $\phi_i(a) \neq \phi_j(a)$. 

$\phi_i$ permutes roots of $p$ in $K$, so $\phi_i(a) = $a root of $p$. 

Therefore $\phi_1(a), dots, \phi_n(a)$ are $n$ distinct roots of $p$ in $K$. 

Thus $[J:H] = n \leq $ number of roots of $p$ in $K \leq \deg p = [L:E] $

For $m > 1$, let $M = E(a_1, \dots, a_{m - 1})$, $L = M(a_m)$. Then $[\ms{G}(E):\ms{G}(M)] \leq [M:E]$, 
$[\ms{G}(M):\ms{G}(L)] \leq [L:M]$.
So $[\ms{G}(E):\ms{G}(L)] \leq [L:E]$

\end{enumerate}

\item 

\begin{enumerate}[label=(\alph*)]

\item $n = [J:H]$, $\phi_1(H),\dots, \phi_n(H)$ are distinct left cosets of $H$ in $J$. 

So $E = \ms{F}(J) \subseteq L = \ms{F}(H)$. 

Suppose $[L:E]> n$. 

Then there exists $a_1, \dots, a_{n + 1} \in L$, linearly independent over $E$. 

There exists $(\phi_i(a_j)) = n\times(n + 1)$ matrix over $K$. 

\end{enumerate}



\end{enumerate}

\section*{Lecture 8, 4/19/23}

We continue the proof


\begin{enumerate}

\item 

\begin{enumerate}[label=(\alph*)]

\item Done

\item Consider the tower $F \subseteq E \subseteq L \subseteq K$, where $[L:E]<\oo$. 

$E = \ms{F}\ms{G}(E)$. 

$[L:E] = [L:\ms{F}\ms{G}(E) \leq [\ms{F}\ms{G}(L):\ms{F}\ms{G}(E)]$

By 1a, $[\ms{G}(E):\ms{G}(L)] \leq [L:E]$. 

By 2a, $[\ms{F}\ms{G}(L):\ms{F}\ms{G}(E)] \leq [\ms{G}(E), \ms{G}(L)]$. 

Thus 
\begin{align*}
[L:E] & \leq [\ms{F}\ms{G}(L): \ms{F}\ms{G}(E)] \\
& \leq [\ms{G}(E): \ms{G}(L)] \\
& \leq [L:E] \\
\end{align*}
So these inequalities are actually all equalities. 

Therefore $[\ms{G}(E):\ms{G}(L)] = [L:E]$. 

$[\ms{F}\ms{G}(L):E] = [\ms{F}\ms{G}(L):\ms{F}\ms{G}(E)] = [L:E]$

By linear algebra, $\ms{F}\ms{G}(L) = L$

\end{enumerate}

\item 

\begin{enumerate}[label=(\alph*)]

\item So we have our $n\times(n + 1)$ matrix $A$ over $K$. We have a linear transformation $K^{n + 1} \to K$, $x \mapsto (Ax^T)^T$

$\ket A \neq \{0\}$

Chose $b \in \ker(A), b \neq 0,$where $b = (b_1, \dots, b_n)$ with $|\{b_i \mid b_i \neq 0\}|$ minimal. 

$\sum_{i=0}^{n + 1}\phi_i(a_j)b_j = 0$ for all $i$.. 

It is okay to permute $j$s, so without loss of generality, $g = (b_1, \dots, b_k, 0, \dots, 0)$ where $b_1, \dots, b_k \neq 0$. 

Without loss of generality, suppose $b_1 = 1$

We claim there exists $\ell$ such that $b_\ell \not\in \ms{F}(J)$.

If all $b_j \in \ms{F}(J)$, then $\phi_i(b_j) = b_j$ for all $i, j$. 

Now for all $i$, $0 = \sum_j\phi_i(a_j)b_j = \sum_j\phi_i(a_j)\phi_i(b_j) = \sum_i(\sum_j a_jb_j)$ 

Thus $\sum_ja_jb_j = 0$ for all $j$, as this is in both $\ms{F}(H),\ms{F}(J)$

This contradicts the linear independence of $a_is$ over $\ms{F}(J)$. 

Thus there exists $\ell\in\{2,\dots, k\}$ such that $b_\ell \not\in\ms{F}(J)$.

Permute $j = 2, \dots, k$ to get $\ell = 2$. 

Now $b = \{1, \underbrace{b_2}_{\not\in\ms{F}(J)}, \dots, b_j, 0, \dots, 0\}$

Thus there exists $\phi \in J$ such that $\phi(b_2) \neq b_2$. 

So $\phi\phi_1(H), \dots, \phi\phi_n(H)$ is another list of distinct left cosets of $H$ in $J$.

Thus there exists $\pi \in S_n$ such that $\phi\phi_j(H) = \phi_{\pi(j)}(H)$ for all $j$. 

$\sum_j\phi_{\pi(i)}(a_j)\phi(b_j) = \sum_j\phi\phi_i(a_j)\phi(b_j) = 0$ for all $i$. 

Chose $b' = (1, \phi(b_2),\dots,\phi(b_k), 0,\dots,0)$ in $\ker A'$, where $A' = (\phi_{\pi(i)}(a_j))$. 

$\ker A' = \ker A$, so $b' \in \ker A$. 

Thus $c = b - b' = (0, \underbrace{b_2 - \phi(b_2)}_{\neq0}, \dots, b_k - \phi(b_k), 0, \dots, 0) \in \ker (A)$

This contradicts the minimality of $k$.

\item Analagous to 1b

\end{enumerate}


\end{enumerate}

\qed

\cor\,
\begin{enumerate}[label=(\alph*)]

\item All finite subgroups of $\Gal(K:F)$ are closed. 

\item $[K:F] <\oo$ implies $|\Gal(K:F)| \leq [K:F]$ is finite

\item $[K:F]<\oo$ implies $\ms{F}$ is injective and $\ms{G}$ is surjective.

\end{enumerate}

\proof

$\{\Id_K\} = \ms{G}(K)$ closed. For all $H \in \Gal(K:F), H = \ms{G}^{-1}\ms{F}(H)$. 

\qed

\prop\,

\begin{enumerate}[label=(\alph*)]

\item $L \in \Intermed$ stable implies that $\Gal(K:L)$ is a normal subgroup of  $\Gal(K:F)$

\item $H$ a normal subgroup of $\Gal(K:F)$ implies $\ms{F}(H) \in \Intermed$ is stable. 

\end{enumerate}

\proof\,

\begin{enumerate}[label=(\alph*)]

\item $\theta\in\Gal(K:L),\phi\in\Gal(K:F)$. $\theta(x) = x$ for all $x \in L$. Since we have stability, $\phi(x) \in L$ for all $x \in L$. 
\begin{align*}
\Rightarrow &\theta\phi(x) = \phi(x) \\
\Rightarrow &\phi^{-1}\theta\phi(x) = x \\
\Rightarrow &\phi^{-1}\theta\phi \in \Gal(K:F) \\
\end{align*}

\item $\phi\in H$, so $\phi(x) = x$ for all $x \in \ms{F}(H)$. Say $\theta\in\Gal(K:F)$. 

By normality of $H$, $\theta^{-1}\phi\theta\in H$. 
\begin{align*}
\Rightarrow & \theta^{-1}\phi\theta(x) = x & \forall x \in \ms{F}(H) \\
\Rightarrow & \phi\theta(x) = \theta(x) & \forall x \in \ms{F}(H),\phi\in H \\
\Rightarrow & \theta(x) \in \ms{F}(H) & \forall x \in \ms{F}(H)
\end{align*}


\end{enumerate}

\section*{Lecture 10, 4/21/23}

\defn

A field extension $F \subseteq K$ is \underline{Galois} if $F = \Fix_K\Gal(K:F)$

\thm[Little Theorem of Galois Theory]

Let $F \subseteq K$ be a field extension, $[K:F]<\oo$. Then the following are equivalent:
\begin{enumerate}

\item $F \subseteq K$ is Galois

\item $\ms{F},\ms{G}$ are inverse bijections

\item $|\Gal(K:F)| = [K:F]$

\end{enumerate}

If so, for all $L \in \Intermed, L \subseteq K$ is Galois.

\proof

$[K:F]<\oo$ implies $\Gal(K:F)$ finite. 

So all $H \in \SubGal$ are closed. 

\subsection*{$1\Rightarrow2$}

Galois $\implies F$ closed. 

So by prop, all $L \in \Intermed$ are closed.


\subsection*{$2\Rightarrow3$}

$F = \ms{F}\ms{G}(F) \implies F$ closed. 

So by prop, 
\begin{align*}
|\Gal(K:F)| & = |\ms{G}(F): \{\Id_K\} ] \\
& = |\ms{G}(F) : \ms{G}(K) ]\\
& = [K:F]
\end{align*}

\subsection*{$3\Rightarrow1$}

$F \subseteq \ms{F}\ms{G}(F) \subseteq K$. 

\begin{align*}
[K:F] & \geq [K:\ms{F}\ms{G}(F) ]\\
& = [\ms{F}(\{\Id_K\}):\ms{F}\ms{G}(F)] \\
\text{by prop } &= [\ms{G}(F):\{\Id_K\}]  \\
\text{by def } & =|\Gal(K:F)| \\
\text{by assumption }& = [K:F] \\
\end{align*}

Thus $[K:F] = [K:\ms{F}\ms{G}(F)]$. 

So $[\ms{F}\ms{G}(F):F] = 1$, i.e. $\ms{F}\ms{G}(F) = F$. 

If we have 1-3, then $L$ closed, so $L = \ms{F}\ms{G}(L) = \Fix_K\Gal(K:L)$. Thus $L \subseteq K$ is Galois.

\qed

\exm

$F \subseteq K, [K:F] = 2, \Char F \neq 2 \implies F \subseteq K$ Galois

Reason: $|\Gal(K:F)| = 2$. 

$F = \Q \subseteq K = \Q(a,\zeta)\subseteq\C$, $a = \sqrt[3]{2}\in\R,\zeta = $root of $X^2 + X + 1$

There exists $\phi_{ij}\in\Gal(K:F)$ such that $\phi_{ij}(a) = a\zeta^i$ ($i = 0, 1, 2$) and $\phi_{ij}(\zeta) = \zeta^j$ ($j = 1, 2$)

Therefore $|\Gal(K:F)| = 6 = [K:F]$. 

For all $L \in \Intermed, L \subseteq K$ is Galois. 

$F \subseteq \Q(a)$ is not Galois. 

\lem

Let $F \subseteq K$ be Galois, $f \in F[X]$ an irreducible polynomial with a root in $K$. Then $f$ splits over $K$ and has no multiple roots. 

\proof

Without loss of generality, suppose $f$ is monic. 

Say $a_i \in K$ is a root of $f$. 

Let $a_1, a_2, \dots, a_r$ be the distinhct roots of $f$ in $K$. 

Set $g = (X - a_1)(X - a_2)\cdots(X - a_r) \in K[X]$. 

We claim $g\in F[X]$. 

Let $\phi\in G = \Gal(K:F)$. 

$\phi$ permutes $\{a_1,\dots, a_r\}$

$\tilde{\phi}(g) = g$ so $\phi(c_i) = c_i$ for all coefficients $c_i$ of $g$. 

Thus all $c_i \in \Fix_K(F) = F_1$, i.e. $g\in F[X]$. 

Let $f = p_{a_1, F}$. 

$g(a_1) = 0 \implies f \mid g$. 

$\deg(g) = r \leq \deg (f)$. 

Thus $f = g$

\qed

\cor 

Let $F \subseteq L \subseteq K$ be a tower fo fields. If $L$ is algebraic over $V$ and Galois over $F$, then $L$ is stable. 

\proof

Chose $a \in L, \phi\in\Gal(K:F)$. $p_{a, F} \in F[X]$ is irreducible, with $a$ as a root. By lemma, $p_{a, F}$ splits over $L$. So all rotos of $p_{a, F}$ in $K$ are in $L$. 

$\phi$ sends $a$ to a root of $p_{a, F}$. Thus $\phi(a) \in L$. 

\thm[Main Theorem of Galois Theory]

Let $F \subseteq K$ be a Galois field extension, $[K:F]<\oo$. Then

\begin{enumerate}

\item There exist inverse bijections
\[
\begin{tikzcd}
\{\text{intermediate fields of }F \subseteq K\} \ar[r, bend right = 10, "\ms{G}"'] & \{\text{subgroups of }\Gal(K:F)\} \ar[l, bend right = 10, "\ms{F}"']
\end{tikzcd}
\]
such that $\ms{G}(L) = \Gal(K:L), \ms{F}(H) = \Fix_K(H)$.

\item If $E \subseteq L$ are intermediate fields, $[\ms{G}(E):\ms{G}(L)] = [L:E]$. 

\item If $H \leq J \leq \Gal(K:L)$, $[\ms{F}(H): \ms{F}(J)] = [J:H]$. 

\item For all intermediate fields $L, L \subseteq K$ is Galois, and 

\[
F \subseteq L\text{ Galois } \iff L\text{ is stable } \iff \Gal(K:L)\text{ is a normal subgroup }\Gal(K:F)
\]
If so, $\Gal(K:F)\Gal(K:L)\cong\Gal(L:F)$

\end{enumerate}

























\end{document}