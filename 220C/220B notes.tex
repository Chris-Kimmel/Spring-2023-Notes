\documentclass[x11names,reqno,14pt]{extarticle}
\input{preamble}
\usepackage[document]{ragged2e}
\usepackage{epsfig}

\pagestyle{fancy}{
	\fancyhead[L]{Spring 2023}
	\fancyhead[C]{220C - Fields}
	\fancyhead[R]{John White}
  
  \fancyfoot[R]{\footnotesize Page \thepage \ of \pageref{LastPage}}
	\fancyfoot[C]{}
	}
\fancypagestyle{firststyle}{
     \fancyhead[L]{}
     \fancyhead[R]{}
     \fancyhead[C]{}
     \renewcommand{\headrulewidth}{0pt}
	\fancyfoot[R]{\footnotesize Page \thepage \ of \pageref{LastPage}}
}
\newcommand{\pmat}[4]{\begin{pmatrix} #1 & #2 \\ #3 & #4 \end{pmatrix}}
\newcommand{\mk}[1]{\mathfrak{#1}}
\newcommand{\into}{\hookrightarrow}
\newcommand{\onto}{\twoheadrightarrow}
\DeclareMathOperator{\Perm}{Perm}

\title{220C - Fields}
\author{John White}
\date{Spring 2023}


\begin{document}

\section*{Lecture 1, 4/3/23}

\defn

A \underline{field} extension $F \subseteq K$ is a field $F$, which is a subfield of a larger field $K$.

One way to keep track of how these are related is the \underline{degree}, $[K:F]$. This is the dimension of $K$ as a vector space over $F$. 

If this degree is $<\oo$, then we refer to this as a \underline{finite extension} (we of course do not mean that they are finite as sets)

If $S \subseteq K$, then $F(S)$ is the subfield of $K$ given by $F \cup S$.

$F[S]$ is the sub-\textit{ring} of $K$ generated by $F \cup S$. These are different in general!

If $S = \{a_1,\dots, a_n\}$, we use $F(a_1, \dots, a_n)$ and $F[a_1,\dots, a_n]$ to denote $F(S)/F[S]$. 

If the extension has the form $F[a]$ for some element $a$, then this is called a \underline{simple extension}. Here, $a$ is called a \underline{primitive element}.

An extension $F \subseteq K$ is called \underline{algebraic} if every $k \in K$ is algebraic over $F$, meaning is the root of some polynomial in $F[x]$

\exm\,

\begin{itemize}

\item $Q \subseteq \R$. This is an infinite extension. Further, it is not an algebraic extension. The hard way to show this is to demonstrate that some element of $\R$ is not algebraic. For example, $\pi,e$ are real, but transcendental over the rationals. 

The easy way is by a simple cardinality argument: Because $\Q$ is countable, $\bar{\Q}$ is, but $\R$ is not

\item $\R\subseteq\C$. This is a finite extension. In fact, it is a simple extension, with primitive $i$.

\item $\Q\subseteq\Q(\sqrt{5})$. This is algebraic. Of course, $\sqrt{5}$ is a root of $x^5 - 1$, but what about the other elements of $\Q(\sqrt{5})$? 

Consider $\{a + b\sqrt{5} \mid a, b \in \Q\}$. This is a subset of $\Q(\sqrt{5})$, a subring, and a subfield: indeed, consider $\frac{1}{a + b\sqrt{5}}$. The ``typical high school trick" is to multiply by the conjugate:
\[
\frac{1}{a + b\sqrt{5}}\frac{a - b\sqrt{5}}{a - b\sqrt{5}} = \frac{a - b\sqrt{5}}{a^2 - 5b^2}
\]
So $\Q(\sqrt{5}) = \{a + b\sqrt{5} \mid a, b \in \Q\}$, as this is a subfield of $Q(\sqrt{5})$ which contains $\sqrt{5}$, so must contain $\Q(\sqrt{5})$. That is, 
\[
\Q(\sqrt{5}) = \{a + b\sqrt{5}\mid a, b \in \Q\} = \Q[\sqrt{5}]
\]
It is easy to see that $[\Q(\sqrt{5}):\Q] = 2$

\end{itemize}

Let $F \subseteq K$ be a field extension, and consider $F[a_1,\dots, a_n]$. 

There exists an evaluation map $\varepsilon:F[X_1,\dots, X_n] \to K$, given by $\varepsilon(f) = f(a_1, \dots, a_n)$. $\varepsilon$ is a ring homomorphism, so $\Im(\varepsilon)$ is a subring of $K$. We have $F[a_1,\dots, a_n] = \Im\varepsilon$

$F(a_1,\dots, a_n)$ is a quotient field for the ring $F[a_1,\dots, a_n]$

Let $F$ be a field, $x, y$ be indeterminants which are independent over $F$. Let $L = F(y)[x]/\langle x^2 - y \rangle$. 

We can check that $x^2 - y$ is irreducible in $F(y)[x]$ because it is quadratic, and $y$ has no square roots. 

So because this is irreducible, $L$ is a field.

In particular, $F(y)$ embeds in $L$ via the natural map $F(y) \into F(y)[x] \onto L$. So $F(y) \subseteq L$. This is a degree two extension of $F(y)$.

\prop

If $[K:F]<\oo$, then $F \subseteq K$ is an algebraic extension.

\proof

Let $n = [K:F]$, and let $a \in K$. Look at $1, a, a^2, \dots, a^n$. This is $n + 1$ elements in $K$, so they must be linearly independent over $F$. So there exists $c_0,c_1,\dots,c_n$, not all zero, such that $\sum_{i=0}^nc_ia^i = 0$. Then $f = \sum_{i=0}^nc_ix^i\in F[x]$ is a polynomial to which $a$ is a solution, so $a$ is algebraic.

\qed

\thm (I)

Let $F \subseteq K$ be a field extension, $a \in K$. Then The Following Are Equivalent (TFAE):

\begin{enumerate}

\item $a$ is algebraic over $F$

\item $\dim_FF[a]<\oo$

\item $[F(a):F] < \oo$

\item $F(a) = F[a]$

\end{enumerate}

\proof

Notice that $3\Rightarrow 2$ are really saying the same thing. Further, $2 + 4 \Rightarrow 3$. So if we can connect 1, 2, 4, then 3 will come along for the ride. Therefore, it is enough to show that $1, 2, 4$ are equivalent.

\subsection*{$1\Rightarrow2$}

There exists a nonzero $f \in F[x]$ such that $f(a) = 0$. $f = \sum_{i=0}^nc_ix^i$, where $c_n\neq0$. So $\sum_{i=0}^nc_ia^i = 0$, and so $a^n = \sum_{i=0}^{n - 1}d_ia^i$, with $d_i \in F$ new coefficients. 

Set $V = \sum_{i=0}^{n - 1}Fa^i$. Then $a^n \in V$. So
\begin{align*}
a^{n + 1} & = \sum_{i=0}^{n - 1}d_ia^{i + 1} \\
			 & = \sum_{j=1}^{n - 1}d_{j - 1}a^j + d_{n - 1}a^n
\end{align*}
But $d_{n - 1}a^n = \sum_{i=0}^{n - 1}d_{n - 1}d_ia^i$. 

Induction gets us that $a^j \in V$ for all $j \geq 0$.

So $V$ is closed under multiplication, hence a subring of $K$.

So $V = F[a]$. Note $\dim_FF[a] = \dim_FV \leq n$, because we used $n$ elements to span in the first place. 

\subsection*{$2\Rightarrow4$}

It will be enough to show $F[a]$ is a field. 

Let $x \in F[a]$, $x\neq0$. Define a map $\mu_x:F[a]\to F[a]$, given by $\mu_x(y) = xy$. This is $F$-linear, and $\ker\mu_x = 0$. We have an injective linear transformation from a finite dimensional vector space to itself, so it has to be an isomorphism onto its image. So there exists $x' \in F[a]$ so that $\mu_x(x') = 1$, so $x$ is invertible.

\section*{Lecture 2, 4/5/23}

We continue the proof.

\subsection*{$4\Rightarrow1$}

If $A = 0$, we are done. If $A \neq 0$, then $\frac{1}{a} \in F(a) = F[a]$. 

So $\frac{1}{a} = \sum_{i=1}^mc_ia^i$ where each $c_i \in F$. Note $1 = \sum_{i=0}^mc_ia^{i + 1}$, so $a$ is a root of $-1 + \sum_{i=0}^mc_ix^{i + 1} = 0$

Thus $a$ is algebraic over $F$. 

\qed

\thm 

Assume $a$ is algebraic over $K$.
\begin{enumerate}[label=(\roman*)]

\item There exists a unique monic polynomial $p \in F[x]$ such that $p(a) = 0$ with minimal degree. We call this the \underline{minimal polynomial} for $a$ over $F$, and write $p_{a, F}$. 

\item $p$ is irreducible.

\item If $g \in F[x]$, $g(a) = 0$, then $p \mid g$ in $F[x]$.

\item $[F(a):F] = \deg p$

\item If $n = \deg p$, then $(1, a, a^2, \dots, a^{n + 1})$ is a basis for $F(a)$ over $F$. 

\item Let $\varepsilon:F[x]\to K, \varepsilon(f) = f(a)$. This induces an isomorphism of rings $\bar{\varepsilon}:\frac{F[x]}{\langle p \rangle} \to F(a), \bar{\varepsilon}(f + \langle p \rangle) = f(a)$

\end{enumerate}

\proof\,

\begin{enumerate}[label=(\roman*)]

\item Since $a$ is algebraic over $F$, there exists $f \in F[x]$ such that $f(a) = 0$. Note that we can divide by the leading coefficient to make $f$ monic with $a$ as a root. Find minimal polynomial of this form, and call it $p$. 

Uniqueness: Suppose $p' \in F[x]$ is monic, $p'(a) = 0$ minimal. Then $(p - p')(a) = 0$. Since $\deg(p - p') < \deg p$, if $p - p' \neq 0$, we have found a monic polynomial with smaller degree than $p$ with $a$ as a root. Contradiction

\item Let $\varepsilon:F[x]\to F(a) = F[a]$ be the evaluation map. $\varepsilon$ induces $\bar{\varepsilon}:\frac{F[x]}{\ker\varepsilon} \to F(a)$. Note $\ker\varepsilon=0$. Since $F[x]$ is a PID, $\ker\varepsilon = \langle q \rangle$ where $0 \neq q \in F[x]$. Without loss of generality, assume $q$ is monic. We know
\begin{itemize}

\item $q$ is irreducible
\item $q(a) = 0$
\item When $g \in F[x], g(a) = 0$, then $q \mid g$

\end{itemize}

Thus, if $g \neq 0$, $\deg (q) \leq \deg (g)$. This implies that $q = p$.

\item See above 

\item $\bar{\varepsilon}$ is also an isomorphism of vectors over $F$. Exercise: If $x \in X + \langle p \rangle$, then $(1, x, x^2, \dots, x^{n + 1})$ is a basis for $\frac{F[x]}{\langle p \rangle}$. Thus $(1, a, a^2, \dots, a^{n - 1})$ is a basis for $F(a)$

Furthermore, $[F(a):F] = n = \deg p$

\qed

Let $F \leq K$ be a field extension, $a \in K$ algebraic over $F$. If $p \in F[x]$ is monic and irreducible with $p(a) = 0$, then $p = p_{a, F}$

\item See iv

\item See ii


\end{enumerate}


\exm

Let $a = \sqrt[4]{5} \in \R_{>0}, p = X^4 - 5 \in \Q[x]$. Since $p$ is irreducible over $\Q[x]$, $p = p_{a, F}$.

Note that $p$ is reducible over $\Q(\sqrt{5})[x]$. In fact, $p_{a, \Q[\sqrt{5}]} = x^2 - \sqrt{5}$. We have the tower of fields:
\[
\begin{tikzcd}
\Q(a) \ar[d, dash] \ar[dd, bend right = 60, "4"'] \\
\Q(\sqrt{5}) \ar[d, dash, "2"] \\
\Q \\
\end{tikzcd}
\]

Let $F \subseteq K \subseteq L$ be a tower of fields. If $a \in L$ is algebraic in $F$, then $a$ is also algebraic over $K$. Furthermore, $p_{a, K} \mid p_{a, F}$ in $K[x]$. 

\prop

If $f \in F[x]$ is a nonzero polynomial of degree $n$, then $f$ has at most $n$ roots in $n$. 

\proof

By induction.

$n = 0:$ trivial. 

$n > 0:$ if there are no roots, we're okay. 

Otherwise, there exists $a \in F$ such that $f(a) = 0$. So $f = (x-alg$, for some $g \in F(x)$. 

$g \neq0, \deg g = n - 1$. Thus $g$ has $\leq n -1 $ roots in $F$. 

Since $\{$roots of $f\} = \{a\} \cup \{$roots of $g\}$, there are $\leq n$ roots of $f$. 

Let $F \subseteq K$ be a field extension. Let $\mc{A} = \{a\in K, a$ algebraic over $F\}$. 

If $F$ is infinite, then $|\mc{A}| = |F|$. If $F$ is finite, $|\mc{A}|$ is countable. 

Let $\mbb{A}$ denote the complex numbers which are algebraic over $\Q$. Note $|\mbb{A}| = |\Q| = \aleph_0$





\section*{Lecture 3, 4/10/23}






















\end{document}