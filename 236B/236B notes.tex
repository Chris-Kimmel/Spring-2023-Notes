\documentclass[x11names,reqno,14pt]{extarticle}
\input{preamble}
\usepackage[document]{ragged2e}
\usepackage{epsfig}

\pagestyle{fancy}{
	\fancyhead[L]{Spring 2023}
	\fancyhead[C]{236B}
	\fancyhead[R]{John White}
  
  \fancyfoot[R]{\footnotesize Page \thepage \ of \pageref{LastPage}}
	\fancyfoot[C]{}
	}
\fancypagestyle{firststyle}{
     \fancyhead[L]{}
     \fancyhead[R]{}
     \fancyhead[C]{}
     \renewcommand{\headrulewidth}{0pt}
	\fancyfoot[R]{\footnotesize Page \thepage \ of \pageref{LastPage}}
}
\newcommand{\pmat}[4]{\begin{pmatrix} #1 & #2 \\ #3 & #4 \end{pmatrix}}
\newcommand{\A}{\mathbb{A}}
\newcommand{\B}{\mathbb{B}}
\newcommand{\fin}{``\in"}
\DeclareMathOperator{\Perm}{Perm}
\DeclareMathOperator{\pdim}{pdim}
\DeclareMathOperator{\gldim}{gldim}
\DeclareMathOperator{\lgldim}{lgldim}
\DeclareMathOperator{\rgldim}{rgldim}
\DeclareMathOperator{\idim}{idim}
\newcommand{\Rmod}{R-\text{mod}}
\newcommand{\RMod}{R-\text{Mod}}
\newcommand{\onto}{\twoheadrightarrow}
\newcommand{\barf}{\bar{f}}
\newcommand{\D}{\mathbb{D}}
\renewcommand{\P}{\mathbb{P}}
\renewcommand{\E}{\mathbb{E}}
\DeclareMathOperator{\Ext}{Ext}
\DeclareMathOperator{\SAB}{SAB}
\DeclareMathOperator{\PAB}{PAB}

\newcommand{\exactlon}[5]{
		\begin{tikzcd}
			0\ar[r]&#1\ar[r,"#2"]& #3 \ar[r,"#4"]& #5 \ar[r]&0
		\end{tikzcd}
}

\title{236B - Homological Algebra}
\author{John White}
\date{Spring 2023}


\begin{document}

\section*{Lecture 1, 4/3/23}

We will begin thinking about homological algebra in a more categorical way. We use "methods of homological algebra," by Gelfond-Manin (Gelfond $\&$ Manin?). Here is a brief overview of some things we will see during the course:

\begin{itemize}

\item Additive and Abelian categories

\item Example: category of sheaves of Abelian groups on a topological space

\item Derived category (objects: complexes of objects in a given Abelian category, morphisms: a quasi-isomorphism of complexes becomes an isomorphism of objects)

\item Derived functors, $f_*, f^*, f_{!}, f^{!}, \D, \cdots$, $\Hom$ for sheaves

\item Triangulated categories

\item $t$-structures

\item Examples: perverse sheaves


\end{itemize} 


Let's remind ourselves of some definitions:

\subsection*{Additive $\&$ Abelian categories}

\exm

\begin{enumerate}[label=(\alph*)]

\item $\Ab$, the category of Abelian groups with group homomorphisms.

\item $\RMod$, the category of $R$-modules, with $R$-module homomorphisms as morphisms.

\item $\SAB, \PAB$, the categories of sheaves of Abelian groups and presheaves of Abelian groups

\item Sheaves of modules over a ringed space 

\item (quasi)-coherent sheaves (ask Zhao)

\end{enumerate}

\defn

An \underline{Abelian category} contains the following information:

\begin{enumerate}

\item Any hom-set $\Hom_{\ms{C}}(X, Y)$ is an Abelian group (+), and the composition of morphisms is bi-additive

In particular:

\begin{itemize}

\item $\Hom_{\ms{C}}$ is a functor $\ms{C}^\circ\times\ms{C}\to\Ab$. We notate the first factor with the $^\circ$

\item $0\in\Hom_{\ms{C}}(X, Y)$ for any objects $X, Y$ of $\ms{C}$

\end{itemize}

\item There exists a zero object $0\in\ms{C}$, that is an object such that $\Hom_{\ms{C}}(0,0) = 0$. 

This gives: $\Hom_{\ms{C}}(0, X) = 0$, $\Hom_{\ms{C}}(X, 0) = 0$ for all objects $X$ of $\ms{C}$.

We know that $\Hom_{\ms{C}}(0, 0)$ consists of one object. In particular, it must be $\Id_0 = 0$. So
\[
(\begin{tikzcd} 0\ar[r, "f"] & X \end{tikzcd} ) = (\begin{tikzcd} 0 \ar[r, "\Id_0"] & 0 \ar[r, "f"] & X \end{tikzcd}) = (\begin{tikzcd} 0\ar[r, "0"] & X \end{tikzcd})
\]
Any two zero objects are isomorphic. 

\item For any $X_1, X_2 \in \ms{C}$, there exists an object $Y$ and morphisms 
\[
\begin{tikzcd} X_1 \ar[r, bend left = 30, "i_1"] &\ar[l, bend left = 20, "p_1"] Y \ar[r, bend right = 30, "p_2"'] & \ar[l, bend right = 20, "i_2"'] X_2 \end{tikzcd}
\]
such that 
\begin{align*}
p_1i_1 & = \Id_{X_1} \\
p_2i_2 & = \Id_{X_2} \\
i_1p_1 + i_2p_2 & = \Id_Y \\
p_2i_1 = p_1i_2 & = 0 \\
\end{align*}

\lem
 We have cartesian diagram 
\[
\begin{tikzcd}
Y'\ar[dr, dotted, "\exists!\varphi"]\ar[drr, "p_1'"]\ar[ddr, "p_2'"'] &  &  \\
& Y \ar[r, "p_1"] \ar[d, "p_2"] & X_1 \ar[d] \\
& X_2 \ar[r] & 0  \\
\end{tikzcd}
\]
That is, for any $Y'$, with morphisms $p_1', p_2'$ as in the diagram, there is a morphism from $Y'$ to $Y$ making the diagram commute. Similarly, we have co-cartesian diagram
\[
\begin{tikzcd}
Y & \ar[l, "i_1"'] X_1 \\
X_2 \ar[u, "i_2"] & 0 \ar[l] \ar[u] \\
\end{tikzcd}
\]

\proof 

We need to construct $\varphi:Y'\to Y$ such that $p_1' = p_1\varphi$ and $p_2'=p_2\varphi$

Take $\varphi = i_1p_1' + i_2p_2'$. Then 
\[
p_1\circ\varphi = \underbrace{p_1i_1}_{=\Id_{X_1}}p_1' + \underbrace{p_1i_2}_{=0}p_2' = p_1'
\]
The uniqueness of $\varphi$ can be verified as an exercise

\qed

\defn

An \underline{additive category} is one which satisfies only the first three of these axioms

\item To state the 4th axiom of Abelian categories, we need more notations:

\defn Let $A_1,A_2$ be objects of $\ms{C}$, and let $\varphi:X\to Y$.

\begin{enumerate}
\item A \underline{kernel} of $\varphi$ is a morphism $i:Z\to X$ such that 
\begin{enumerate}[label=(\alph*)]

\item $\varphi\circ i = 0$

\item For all $i':Z'\to X$ such that $\varphi\circ i' = 0$, there is a unique $g:Z'\to Z$ such that $i'=i\circ g$. 
\[
\begin{tikzcd}
 & Z' \ar[d, "i'"] \ar[dr, "0"] \ar[dl, dotted, "g"'] & \\
Z \ar[r, "i"] & X \ar[r, "\varphi"] & Y \\
\end{tikzcd}
\]

\end{enumerate}

\item A \underline{cokernel} is a kernel but with the arrows reversed

\end{enumerate}

Exercise: Verify that 1 is equivalent to the following: for all $Z'\in\ms{C}$, 
\[
\begin{tikzcd}
0 \ar[r]& \Hom(Z',Z) \ar[r, "i_*"] & \Hom(Z', X) \ar[r, "\varphi_*"] & \Hom(Z', Y) 
\end{tikzcd}
\]
is exact, and similarly for cokernel

With all that, we are ready for: 

\item For any $\varphi:X\to Y$, there exists a sequence of morphisms 
\[
\begin{tikzcd}
K\ar[r, "k"] & X \ar[r, "i"] & I \ar[r, "j"] & Y \ar[r, "c"] & K'
\end{tikzcd}
\]
such that
\begin{enumerate}[label=(\alph*)]

\item $j\circ i = \varphi$

\item $k = \ker \varphi, k' = \coker\varphi$

\item $I = \coker k = \ker c$

\end{enumerate}

\end{enumerate}

This finishes the definition












\end{document}
