\documentclass[x11names,reqno,14pt]{extarticle}
\input{preamble}
\usepackage[document]{ragged2e}
\usepackage{epsfig}

\pagestyle{fancy}{
	\fancyhead[L]{Spring 2023}
	\fancyhead[C]{236B}
	\fancyhead[R]{John White}
  
  \fancyfoot[R]{\footnotesize Page \thepage \ of \pageref{LastPage}}
	\fancyfoot[C]{}
	}
\fancypagestyle{firststyle}{
     \fancyhead[L]{}
     \fancyhead[R]{}
     \fancyhead[C]{}
     \renewcommand{\headrulewidth}{0pt}
	\fancyfoot[R]{\footnotesize Page \thepage \ of \pageref{LastPage}}
}
\newcommand{\pmat}[4]{\begin{pmatrix} #1 & #2 \\ #3 & #4 \end{pmatrix}}
\newcommand{\A}{\mathbb{A}}
\newcommand{\B}{\mathbb{B}}
\newcommand{\fin}{``\in"}
\DeclareMathOperator{\Perm}{Perm}
\DeclareMathOperator{\pdim}{pdim}
\DeclareMathOperator{\gldim}{gldim}
\DeclareMathOperator{\lgldim}{lgldim}
\DeclareMathOperator{\rgldim}{rgldim}
\DeclareMathOperator{\idim}{idim}
\DeclareMathOperator{\supp}{supp}
\newcommand{\Rmod}{R-\text{mod}}
\newcommand{\RMod}{R-\text{Mod}}
\newcommand{\onto}{\twoheadrightarrow}
\newcommand{\barf}{\bar{f}}
\newcommand{\D}{\mathbb{D}}
\newcommand{\into}{\hookrightarrow}
\renewcommand{\P}{\mathbb{P}}
\renewcommand{\E}{\mathbb{E}}
\DeclareMathOperator{\Ext}{Ext}
\DeclareMathOperator{\SAb}{SAb}
\DeclareMathOperator{\PAb}{PAb}
\DeclareMathOperator{\Ob}{Ob}

\newcommand{\exactlon}[5]{
		\begin{tikzcd}
			0\ar[r]&#1\ar[r,"#2"]& #3 \ar[r,"#4"]& #5 \ar[r]&0
		\end{tikzcd}
}

\title{236B - Homological Algebra}
\author{John White}
\date{Spring 2023}


\begin{document}

\section*{Lecture 1, 4/3/23}

We will begin thinking about homological algebra in a more categorical way. We use "methods of homological algebra," by Gelfond-Manin (Gelfond $\&$ Manin?). Here is a brief overview of some things we will see during the course:

\begin{itemize}

\item Additive and Abelian categories

\item Example: category of sheaves of Abelian groups on a topological space

\item Derived category (objects: complexes of objects in a given Abelian category, morphisms: a quasi-isomorphism of complexes becomes an isomorphism of objects)

\item Derived functors, $f_*, f^*, f_{!}, f^{!}, \D, \cdots$, $\Hom$ for sheaves

\item Triangulated categories

\item $t$-structures

\item Examples: perverse sheaves


\end{itemize} 


Let's remind ourselves of some definitions:

\subsection*{Additive $\&$ Abelian categories}

\exm

\begin{enumerate}[label=(\alph*)]

\item $\Ab$, the category of Abelian groups with group homomorphisms.

\item $\RMod$, the category of $R$-modules, with $R$-module homomorphisms as morphisms.

\item $\SAb, \PAb$, the categories of sheaves of Abelian groups and presheaves of Abelian groups

\item Sheaves of modules over a ringed space 

\item (quasi)-coherent sheaves (ask Zhao)

\end{enumerate}

\defn

An \underline{Abelian category} contains the following information:

\begin{enumerate}

\item Any hom-set $\Hom_{\ms{C}}(X, Y)$ is an Abelian group (+), and the composition of morphisms is bi-additive

In particular:

\begin{itemize}

\item $\Hom_{\ms{C}}$ is a functor $\ms{C}^\circ\times\ms{C}\to\Ab$. We notate the first factor with the $^\circ$

\item $0\in\Hom_{\ms{C}}(X, Y)$ for any objects $X, Y$ of $\ms{C}$

\end{itemize}

\item There exists a zero object $0\in\ms{C}$, that is an object such that $\Hom_{\ms{C}}(0,0) = 0$. 

This gives: $\Hom_{\ms{C}}(0, X) = 0$, $\Hom_{\ms{C}}(X, 0) = 0$ for all objects $X$ of $\ms{C}$.

We know that $\Hom_{\ms{C}}(0, 0)$ consists of one object. In particular, it must be $\Id_0 = 0$. So
\[
(\begin{tikzcd} 0\ar[r, "f"] & X \end{tikzcd} ) = (\begin{tikzcd} 0 \ar[r, "\Id_0"] & 0 \ar[r, "f"] & X \end{tikzcd}) = (\begin{tikzcd} 0\ar[r, "0"] & X \end{tikzcd})
\]
Any two zero objects are isomorphic. 

\item For any $X_1, X_2 \in \ms{C}$, there exists an object $Y$ and morphisms 
\[
\begin{tikzcd} X_1 \ar[r, bend left = 30, "i_1"] &\ar[l, bend left = 20, "p_1"] Y \ar[r, bend right = 30, "p_2"'] & \ar[l, bend right = 20, "i_2"'] X_2 \end{tikzcd}
\]
such that 
\begin{align*}
p_1i_1 & = \Id_{X_1} \\
p_2i_2 & = \Id_{X_2} \\
i_1p_1 + i_2p_2 & = \Id_Y \\
p_2i_1 = p_1i_2 & = 0 \\
\end{align*}

\lem
 We have cartesian diagram 
\[
\begin{tikzcd}
Y'\ar[dr, dotted, "\exists!\varphi"]\ar[drr, "p_1'"]\ar[ddr, "p_2'"'] &  &  \\
& Y \ar[r, "p_1"] \ar[d, "p_2"] & X_1 \ar[d] \\
& X_2 \ar[r] & 0  \\
\end{tikzcd}
\]
That is, for any $Y'$, with morphisms $p_1', p_2'$ as in the diagram, there is a morphism from $Y'$ to $Y$ making the diagram commute. Similarly, we have co-cartesian diagram
\[
\begin{tikzcd}
Y & \ar[l, "i_1"'] X_1 \\
X_2 \ar[u, "i_2"] & 0 \ar[l] \ar[u] \\
\end{tikzcd}
\]

\proof 

We need to construct $\varphi:Y'\to Y$ such that $p_1' = p_1\varphi$ and $p_2'=p_2\varphi$

Take $\varphi = i_1p_1' + i_2p_2'$. Then 
\[
p_1\circ\varphi = \underbrace{p_1i_1}_{=\Id_{X_1}}p_1' + \underbrace{p_1i_2}_{=0}p_2' = p_1'
\]
The uniqueness of $\varphi$ can be verified as an exercise

\qed

\defn

An \underline{additive category} is one which satisfies only the first three of these axioms

\item To state the 4th axiom of Abelian categories, we need more notations:

\defn Let $A_1,A_2$ be objects of $\ms{C}$, and let $\varphi:X\to Y$.

\begin{enumerate}
\item A \underline{kernel} of $\varphi$ is a morphism $i:Z\to X$ such that 
\begin{enumerate}[label=(\alph*)]

\item $\varphi\circ i = 0$

\item For all $i':Z'\to X$ such that $\varphi\circ i' = 0$, there is a unique $g:Z'\to Z$ such that $i'=i\circ g$. 
\[
\begin{tikzcd}
 & Z' \ar[d, "i'"] \ar[dr, "0"] \ar[dl, dotted, "g"'] & \\
Z \ar[r, "i"] & X \ar[r, "\varphi"] & Y \\
\end{tikzcd}
\]

\end{enumerate}

\item A \underline{cokernel} is a kernel but with the arrows reversed

\end{enumerate}

Exercise: Verify that 1 is equivalent to the following: for all $Z'\in\ms{C}$, 
\[
\begin{tikzcd}
0 \ar[r]& \Hom(Z',Z) \ar[r, "i_*"] & \Hom(Z', X) \ar[r, "\varphi_*"] & \Hom(Z', Y) 
\end{tikzcd}
\]
is exact, and similarly for cokernel

With all that, we are ready for: 

\item For any $\varphi:X\to Y$, there exists a sequence of morphisms 
\[
\begin{tikzcd}
K\ar[r, "k"] & X \ar[r, "i"] & I \ar[r, "j"] & Y \ar[r, "c"] & K'
\end{tikzcd}
\]
such that
\begin{enumerate}[label=(\alph*)]

\item $j\circ i = \varphi$

\item $k = \ker \varphi, k' = \coker\varphi$

\item $I = \coker k = \ker c$

\end{enumerate}

\end{enumerate}

This finishes the definition

\section*{Lecture 2, 4/5/23}

\subsection*{Sheaves}

Here are some examples of sheaves from complex analysis

\exm\,

\begin{enumerate}[label=(\alph*)]

\item The set of holomorphic functions on $\mbb{P} = \C \sup \{\oo\}$

For each open subset $U$ of $\mbb{P}$, we can consider the ring of holomorphic function $f:U\to\C$, $\ms{H}(U)$. 

The collection of $\{(f, V) \mid V \text{ open}, f \in \ms{H}(V) \}$ is called the sheaf $\ms{O}$ of holomorphic functions on $\mbb{P}$.

\item The sheaf of solutions of a linear ODE

Let $U \subseteq \mbb{P}$ be open, and let $a_i(z) \in \Gamma(U,\ms{O})$ (in this context this will wind up meaning $\ms{H}(U)$), $i = 0,1,\dots,n - 1$. 

Denote by $S$ the collection of $(V, f)$ such that $V \subseteq U$ is open, and $f$ is holomorphic in $V$ such that 
\[
Lf \eqdef \frac{d^nf}{dz^n} + \sum_{i=0}^{n - 1}a_i(z)\frac{d^if}{dz^i} = 0
\]

Let $\Gamma(V) = \{f \in \ms{H}(V) \mid Lf = 0 \}$. When $V$ is connected and simply connected, it is a basic result of ODEs that $\Gamma(V)\cong \C^n$

In general, it may have to do with the topology of $V$. For example, if $U = \C\setminus\{0\}$, $L = \frac{d^2}{dz^2} + \frac{1}{z}\frac{d}{dz}$, the solutions are $c_1\log(z) + c_2$ for any ``branch" of $\log(z)$, but $\Gamma(V) = \{\text{constant}\}$. This is related to the Riemann-Hilbert correspondence (whatever that is)

\end{enumerate}

\defn\,

\begin{enumerate}[label=(\alph*)]

\item A \underline{presheaf of sets $\mc{F}$ on a topological space $Y$} consists of the following data:

\begin{itemize}

\item A set $\mc{F}(U)$ for any open $U \subseteq Y$

\item For any open $V \subseteq U$, a (restriction) map $\gamma_{U,V}:\mc{F}(U)\to\mc{F}(V)$ such that $\gamma_{V,V} = \Id_{\mc{F}(V)}$, and if $W \subseteq V \subseteq U$, then $\gamma_{V,W}\circ\gamma_{U,V} = \gamma_{U,W}$. When there is ambiguity about which sheaf $\gamma$ belongs to, we further specify with $\gamma^{\mc{F}}$

\end{itemize}

\item A presheaf $\mc{F}$ is a \underline{sheaf} if: 

\begin{itemize}

\item For any open covering $U = \cup_{i\in I}U_i$ and $s_i \in \mc{F}(U_i)$ such that for all $i, j \in I$, 
\[
\gamma_{U_i, U_i \cap U_j}(s_i) = \gamma_{U_j, U_i \cap U_j}(s_j)
\]
then there exists a unique $s\in \mc{F}(U)$ such that $s_i = \gamma_{U,U_i}(s)$ for all $i$. 

\end{itemize}

\item A \underline{morphism of presheaves} $f:\mc{F}\to\mc{G}$ is a family of maps $f(U):\mc{F}(U)\to\mc{G}(U)$ for all open $U \subseteq Y$, such that for all open $V \subseteq U$, the diagram
\[
\begin{tikzcd}
\mc{F}(U) \ar[d, "\gamma^{\mc{F}}_{U,V}"'] \ar[r, "f(U)"] & \mc{G}(U) \ar[d, "\gamma^{\mc{G}}_{U, V}"] \\
\mc{F}(V) \ar[r, "f(V)"] & \mc{G}(V) \\
\end{tikzcd}
\]
commutes.

A \underline{morphism of sheaves} is a morphism of the underlying presheaves



\end{enumerate}

A presheaf $\mc{F}$ of groups/rings/$\mbb{F}$-vector spaces is a presheaf such that $\mc{F}(U)$ is a group/ring/$\mbb{F}$-vector space. Then $\gamma_{U,V}$ is a morphism of groups/rings/$\mbb{F}$-vector spaces. 

\,

Let $\mc{F},\mc{G}$ be two Abelian presheaves (meaning presheaves of Abelian groups), and let $f:\mc{F}\to\mc{G}$ be a morphism of Abelian presheaves. 

Let $K(U) = \ker(f(U))$, $C(U) = \coker(f(U))$ with natural restrictions.

\defn

A sequence of presheaves
\[
\begin{tikzcd}
\mc{F}\ar[r, "f"] & \mc{G} \ar[r, "g"] & \mc{H}
\end{tikzcd}
\]
is \underline{exact} if, for all open $U$, 
\[
\begin{tikzcd}
\mc{F}(U) \ar[r, "f(U)"] & \mc{G}(U) \ar[r, "g(U)"] & \mc{H}(U)
\end{tikzcd}
\]
is exact. 

So, presheaves of Abelian groups, with morphisms of Abelian presheaves, is an Abelian category.

How about Abelian sheaves? 

Let $f:\mc{F}\to\mc{G}$ be a morphism of Abelian sheaves. Once again, let $K(U) = \ker(f(U)), C(U) = \coker(f(U)) = \frac{\mc{G}(U)}{f(U)(\mc{F}(U))}$. 

\prop\,

\begin{enumerate}[label=(\alph*)]

\item The kernel $K$ is an Abelian sheaf

\item The cokernel $C$ is always a presheaf, but might not be a sheaf.


\end{enumerate}

\section*{Lecture 3, 4/7/23}

\proof\,

\begin{enumerate}[label=(\alph*)]

\item Let $U = \bigcup U_i$, $s_i \in K(U_i)$ agree on pairwise intersections. As $K(U_i) \into \mc{F}(U_i)$, there exists a unique $s \in \mc{F}(U)$ such that $s|_{U_i} = s_i \in \mc{F}(U_i)$ for all $i$.

Note that $f(s)|_{U_i} = f(s|_{U_i}) = 0$, so $f(s) = 0 \in \mc{G}(U)$. Here, we are using the uniqueness of gluing in $\mc{G}$.  

Then $s \fin K(U)$

\item First, we give an example of a cokernel which is a presheaf, but not a sheaf. Let $Y = \C\setminus\{0\}$, $f:\mc{O}_Y\to\mc{O}_Y$ given by $\varphi\mapsto\frac{d\varphi}{d\zeta}$. 

\begin{itemize}

\item For any $y \in Y$, there exists a neighborhood $V_y \ni y$ such that $\coker f(V_y) = 0$. That is, for every $f\in \mc{H}(V_y)$, there is a $g \in \mc{H}(V_y)$ so that $\frac{dg}{d\zeta} = f$ in $V_y$ (every point admits a simply connected neighborhood)

\item However, $\coker f(Y) \cong \C$: $\Psi = \sum_{i=-\oo}^\oo a_i\zeta^i = \frac{d\varphi}{d\zeta}$ has a solution iff $a_{-1} = 0$. This is because $\frac{1}{z}$ is defined on $Y$. (keyword: vanishing cycle and v-filtration). 

Hence this violates the uniqueness: any $\bar{\Psi} \in \coker f(Y)$ restricts to $0 \in \coker f(V_y)$. However, the $V_y$ cover $Y$. So, we have an open cover, with sections of each, which agree on intersections, but which do not have a \underline{unique} global section to which they all glue. So, this is not a sheaf.

\end{itemize}

\end{enumerate}

\qed

\subsection*{Sheafification}

Denote by $\SAb$ the additive category of abelian sheaves on a fixed topological space $M$. We have the inclusion functor $\iota:\SAb\to\PAb$. 

\prop

$\iota$ admits a left adjoint $s:\PAb\to\SAb$, i.e. 
\[
\Hom_{\SAb}(sX, Y) \cong \Hom_{\PAb}(X, \iota Y)
\]
and this isomorphism is natural in both $X$ and $Y$.

\proof

Let 
\[
s(X)(U) \eqdef \frac{\{(\{U_i\}_{i\in I}, e_i) \mid U = \bigcup_{i\in I} U_i, e_i \in X(U_i), e_i|_{U_i \cap U_j} = e_j|_{U_i \cap U_j}\}}{\sim}
\]
Where $(\{U_i\}, e_i) \sim (\{U_j'\}, e_j)$ if there exists $\{U_k''\}$ refining $\{U_i\}, \{U_j'\}$ and $e_i|_{U_k''} = e_j'|_{U_k''}$ for $U_k'' \subset U_i \cap U_j'$. 

Define $\gamma_{U,V}:sX(U)\to sX(V)$, for $V \subseteq U$, by 
\[
\gamma_{U,V}[(\{U_i\},e_i)] = [\{U_i \cap V\}, e_i|_{U_i \cap V}]
\]

There is a lot to verify; see [GM] 2.5.13

\qed

\exm

For $\coker f$ from previous example, now $[\bar{\Psi}] = [0]$, since, when we restrict to $V_y$, $\bar{\Psi}$ becomes 0. So $\coker f = 0$

\,

With this modification, $\SAb$ is an abelian category!

\prop

Let $\varphi:X\to Y$ be a morphism of abelian sheaves, and let 
\[
\begin{tikzcd}
K\ar[r, "k"] & \iota X \ar[r, "i"] & I \ar[r, "j"] & \iota Y \ar[r, "c"] & K'
\end{tikzcd}
\]
be the canonical decomposition of $\iota(\varphi)$ in (the abelian category!) $\PAb$. Then
\[
\begin{tikzcd}
sK(= K)\ar[r, "sk"] & X (= s\iota X) \ar[r, "si"] & sI \ar[r, "sj"] & Y (= s\iota Y) \ar[r, "sc"] & sK'
\end{tikzcd}
\]
is the canonical decomposition of $\varphi$ in $\SAb$. In particular, $\SAb$ is an abelian category. 

\proof

We'll just verify that $sK'$ is indeed the cokernel:

Let $Z\in \Ob\SAb$. Then there exists an exact sequence
\[
\exactshort{\Hom_{\PAb}(K',\iota Z)}{}{\Hom_{\PAb}(Y,\iota Z)}{}{\Hom_{\PAb}(X,\iota Z)}
\]
(this is equivalent to saying $K'$ is $\coker\varphi$ in $\PAb$). By adjunction,
\[
\exactshort{\Hom_{\SAb}(sK', Z)}{}{\Hom_{SAb}(\underbrace{sY}_{=Y}, Z)}{}{\Hom_{\SAb}(sX,Z)}
\]
is exact. Other parts are exercises

\qed

\defn

Let $A$ be an abelian group, $Y$ a topological space. 

\begin{enumerate}[label=(\alph*)]

\item The \underline{constant presheaf $\A$ on $Y$} is $\A(U) = A$ for all open $U \subseteq Y$, and $\gamma_{U,V} = \Id_A$ for any open $V \subseteq U$. 

\item The \underline{constant sheaf $\mc{A}$ on Y} is $s\A$.

(Check: for connected $U$, $\mc{A}(U) = A$)

\item A sheaf $\mc{F}$ is \underline{locally constant} if any point has a neighborhood $U$ such that $\mc{F}|_U$ is a constant sheaf. (for open $V \subset U \subset Y$, $\mc{F}|_U(V) \eqdef \mc{F}(V)$) (keyword: representation of $\pi_1$ and local systems)

\end{enumerate}

\subsection*{Germs and stalks}

\defn

The \underline{stalk of a (pre)sheaf at a point $y$} is
\[
\mc{F}_y \eqdef \lim_{\longrightarrow\atop{V\ni y}}\mc{F}(V)
\]
with the limit taken with respect to inclusion.

Concretely, $\mc{F}_y \eqdef \frac{\{(s, V) \mid y \in V, s \in \mc{F}(V)\}}{\sim}$, where $(s, V) \sim (s', V')$ if there exists a $W \subseteq V \cap V'$ such that $\gamma_{V,W}(s) = \gamma_{V',W}(s')$. 

Such an equivalence class is called a \underline{germ}

\rem [GM, I.5.5, I.5.6]

If we have an exact sequence of sheaves
\[
\exactshort{\mc{F}'}{}{\mc{F}}{}{\mc{F}''}
\]
if and only if for any $y \in Y$, 
\[
\exactshort{\mc{F}_y'}{}{\mc{F}_y}{}{\mc{F}_y''}
\]
is exact. This can be verified as an extremely important exercise.

\defn

For $s \in \mc{F}(U)$, the \underline{support of $s$}, $\supp s$, is the closure of the set of points at which the germ of $s$ is not zero. 

\rem 

In the definition of a stalk, we can replace a point $y$ by a closed subset $Z$ of $Y$.














\end{document}
